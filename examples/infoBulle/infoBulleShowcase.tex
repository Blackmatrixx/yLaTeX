\documentclass{article}

\usepackage{lipsum}

\usepackage{infoBulle}

\usepackage[%
	a4paper,
%	showframe,
	twoside,
	top=27mm,
	bottom=27mm,
	inner=20mm,
	outer=20mm,
	ignorehead,
	ignorefoot,
	ignoremp,
	marginparwidth=0mm,
	marginparsep=0mm,
	headsep=7mm,
	footskip=14mm,
	headheight=12.2pt,
]{geometry}
\setlength{\marginparpush}{\baselineskip}

\title{infoBulle Showcase}
\author{Harvey Sheppard}

%\infoBulleConfiguration{logo/shape = hexagon}

\begin{document}
	
	\maketitle
	
	\section{Text InfoBulles}
	
	\infoInfo{\lipsum[2]}{\lipsum[2]}
	
	\questionInfo{Is this a question?}{\lipsum[2]}
	
	\checkInfo{What an Example}{\lipsum[2]}
	
	\warningInfo{Warning!}{\lipsum[2]\lipsum[2]}
	
	\criticalInfo{Critical Info!}{\lipsum[2]}
	
	\tipsInfo{Tips!}{\lipsum[2]}
	
	\begin{CodeInfo}{Some code!}[And a caption]
		\begin{CodeInfoLst}
#!/usr/bin/env python
print "Hello World"
		\end{CodeInfoLst}
		\lipsum[2]
	\end{CodeInfo}
	
	\section{Math InfoBulles}
	
	\begin{theorem}[Pythagoras]
		Assuming we have a rectangle triangle with the hypotenuse named c. Then:
		\[
		a^2 + b^2 = c^2
		\]
	\end{theorem}
	
	\begin{lemma}[Pythagoras]
		Assuming we have a rectangle triangle with the hypotenuse named c. Then:
		\[
		a^2 + b^2 = c^2
		\]
	\end{lemma}
	
	\begin{axiom}[Pythagoras]
		Assuming we have a rectangle triangle with the hypotenuse named c. Then:
		\[
		a^2 + b^2 = c^2
		\]
	\end{axiom}
	
	\begin{proposition}[Pythagoras]
		Assuming we have a rectangle triangle with the hypotenuse named c. Then:
		\[
		a^2 + b^2 = c^2
		\]
	\end{proposition}
	
	\begin{definition}[Pythagoras]
		Assuming we have a rectangle triangle with the hypotenuse named c. Then:
		\[
		a^2 + b^2 = c^2
		\]
	\end{definition}
	
	\begin{corollary}[Pythagoras]
		Assuming we have a rectangle triangle with the hypotenuse named c. Then:
		\[
		a^2 + b^2 = c^2
		\]
	\end{corollary}
	
	
\end{document}
