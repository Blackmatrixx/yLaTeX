% !TeX spellcheck = en_US
\documentclass{yLectureNote}

\title{Analyse I}
\subtitle{subtitle}
\author{Yves Zumbach}
\date{Fall Semester 2016}
\yLanguage{English}

\professor{M. Professor}


\usepackage{lipsum}


\begin{document}
	\titleOne
	
	\yTableOfContent[\vfill\hfill\authorBlock{
		\authorName{Yves Zumbach}
		\authorAdressLineOne{EPFL}
		\authorAdressLineTwo{Route cantonale}
		\authorAdressLineThree{Lausanne, Vaud, Suisse}
	}]
	
	
	
	\chapter{Mathematical Argumentation and Proofs}
	\printMarginPartialToc
	
	
	\section{Mathematical Argumentation}
	\classDate{13}{9}{2016}
	\nextExerciseSet
	
	\lipsum[1]
	
	\sideTips{Don't remember to practice your math! It is the only way to get through!}
	
	
	\subsection{Using Conditions}
	\lipsum[2]
	\sideCritical{I should reread this part, as I didn't understood it well... Maybe it's because it's in Latin?}
	
	\lipsum[3]\sideNote{\sideTitle{Title} \lipsum*[7]}
	
	\checkInfo{Yeaaaah!}{\lipsum[4]}
	
	
	\section{Mathematical Proofs}
	\begin{theorem}[Pythagoras]
		Assuming we have a rectangle triangle with the hypotenuse named c. Then:
		\[
			a^2 + b^2 = c^2
		\]
	\end{theorem}
	
	
	\sideInfo{This formula is really important:
		\begin{equation}
		a^2 + b^2 = c^2
		\end{equation}
	}
	
	\yOrnament
	
\end{document}