
\printMiniToc

\section{Diffusion}
Si \nomJeu\ vient un jour à être distribué ce sera gratuitement et librement, bien sûr! Chacun pourra télécharger le jeu et l'essayer sans payer. Mais plus encore, chacun sera libre de télécharger et modifier son contenu puis de le redistribuer (sous la même licence). Il pourrait même être envisageable de rendre le jeu public avant qu'il ne soit terminé et demander de l'aide à la communauté pour le terminer mais cela implique beaucoup de coordination et une certaine perte de contrôle sur la version \textit{officiel} du jeu.


\section{Réaliser un jeu vidéo libre?}
Pour la réalisation de ce jeu, je m'étais fixé comme condition de n'utiliser que des logiciels gratuits. Atteindre cet objectif fut relativement simple. Il existe, de par le Net, une multitude de programmes gratuits et c'est sans difficulté que j'ai trouvé les outils qui m'étaient nécessaires.

Pour ce qui est du deuxième objectif, à savoir utiliser le plus possible des logiciels libres, je fus agréablement surpris de constater qu'ils sont également nombreux et divers. En effet, il existe, pour la plupart des tâches que nécessite la création d'un jeu, des outils open source. Un seul programme a fait exception à la règle pour \nomJeu: Sketchup. J'avais besoin de ce dernier pour convertir les modèles téléchargés d'Internet vers des formats plus aisément éditables. Impossible de s'en passer donc et impossible de trouver une alternative. Mais ce n'est qu'une goutte au milieu de l'océan.

À mon goût, les objectifs fixés au début de ce document sont atteints. C'est pour moi la preuve qu'Internet pourrait apporter une révolution dans notre mode de fonctionner: un monde où les gens travailleraient pour la communauté, partageraient et amélioreraient sans attendre de contre-partie, où l'information serait libre et gratuite et où les outils seraient disponibles pour tous ceux qui en auraient besoin.


\section{Réaliser un jeu vidéo seul?}
L'expérience fut moins concluante de ce point de vue. S'il était évident que je ne pourrais finir un jeu complet en moins d'une année, je suis cependant un peu déçut de n'avoir pu mener ce projet plus loin, dans le temps imparti. Certaines tâches qui me semblaient aisée ont parfois demandés des heures de recherches et certains \anglicisme{bugs} m'ont coûtés un temps énorme, précieux, passé à tenter de comprendre la source du problème. C'est donc une esquisse un peu moins avancée que ce que j'espérais que je rends.

Le Travail de Maturité se termine donc, mais pas le projet \nomJeu. Je n'ai aucune raison de m'arrêter maintenant; j'ai déjà un scénario et les bases du jeu. Il ne me reste \enquote{plus} qu'à finir ce qui est déjà débuté. Et qui sait, dans quelques années \nomJeu\ sera peut-être enfin prêt.


\section{Bilan personnel}
Pour être honnête, ce Travail de Maturité fut pour moi une source de stress; j'aurais voulu porter la réalisation du jeu plus loin, ajouter plus de détails, de fonctionnalité. S'engager dans un travail d'une telle ampleur avec la volonté de le mener à bout peut causer des tensions considérable.

Mais il est évident que ce Travail m'a aussi beaucoup apporté. J'ai dû persévérer malgré les difficultés et parfois l'énervement. Ce fut l'occasion d'acquérir toutes sortes de connaissances techniques: modélisation, programmation, édition de son, création de cartes, etc. J'ai aussi découvert l'univers du game designer, je suis en quelque sorte passé \enquote{de l'autre côté de l'écran}. 


