% !TeX spellcheck = en_GB
%%%%%%%%%%%%%%%%%%%%%%%%%%%%%%%%%%%%%%
%		Basic configuration
%%%%%%%%%%%%%%%%%%%%%%%%%%%%%%%%%%%%%%
\documentclass[a4paper, 11pt, oneside, fleqn]{article}
\usepackage[no-math]{fontspec}

\usepackage{polyglossia}
\setdefaultlanguage{french}
\setotherlanguages{english}



%%%%%%%%%%%%%%%%%%%%%%%%%%%%%%%%%%%%%%
%		Various packages
%%%%%%%%%%%%%%%%%%%%%%%%%%%%%%%%%%%%%%
\usepackage{metalogo}
\usepackage{microtype}
\usepackage{graphicx}
\usepackage{wrapfig}
\usepackage[german=swiss]{csquotes}
\usepackage{calc}
\usepackage[usenames,dvipsnames,svgnames,table]{xcolor}
\usepackage{amsmath, amsfonts, amssymb}
\usepackage{appendix}
\usepackage{setspace}
\usepackage{multicol}
\usepackage{listings}
\lstset{
	basicstyle=\ttfamily,
	columns=fullflexible,
	keepspaces=true,
	basicstyle=\ttfamily\small,
}



%%%%%%%%%%%%%%%%%%%%%%%%%%%%%%%%%%%%%%
%		Colors
%%%%%%%%%%%%%%%%%%%%%%%%%%%%%%%%%%%%%%
\definecolor{mainColor}{RGB}{211, 47, 47}

\newcommand{\inColor}[1]{{\bfseries\color{mainColor}#1}}



%%%%%%%%%%%%%%%%%%%%%%%%%%%%%%%%%%%%%%
%		Layout
%%%%%%%%%%%%%%%%%%%%%%%%%%%%%%%%%%%%%%
\usepackage[
	xetex,
	a4paper,
	ignoreheadfoot,
	left=2cm,
	right=2cm,
	top=2cm,
	bottom=2cm,
	nohead,
	marginparwidth=0cm,
	marginparsep=0mm
]{geometry}
\setlength{\skip\footins}{1cm}
\setlength{\footnotesep}{2mm}
\setlength{\parskip}{1ex}
\setlength{\parindent}{0ex}
\setlength{\columnsep}{15mm}



%%%%%%%%%%%%%%%%%%%%%%%%%%%%%%%%%%%%%%
%		Tables
%%%%%%%%%%%%%%%%%%%%%%%%%%%%%%%%%%%%%%
\usepackage{array}
\usepackage{tabu}
\usepackage{longtable}

\definecolor{tableLineOne}{RGB}{245, 245, 245}
\definecolor{tableLineTwo}{RGB}{224, 224, 224}



%%%%%%%%%%%%%%%%%%%%%%%%%%%%%%%%%%%%%%
%		Links
%%%%%%%%%%%%%%%%%%%%%%%%%%%%%%%%%%%%%%
\usepackage{hyperref}
\hypersetup{
	pdftoolbar=false,
	pdfmenubar=true,
	pdffitwindow=false,
	pdfborder={1 1 0},
	pdfcreator=LaTeX,
	colorlinks=true,
	linkcolor=blue,
	linktoc=all,
	urlcolor=blue,
	citecolor=blue,
	filecolor=blue
}



%%%%%%%%%%%%%%%%%%%%%%%%%%%%%%%%%%%%%%
%		Itemize and consort
%%%%%%%%%%%%%%%%%%%%%%%%%%%%%%%%%%%%%%
\def\labelitemi{---}
\usepackage{enumitem}
\setlist[itemize]{nosep}
\setlist[description]{nosep}
\setlist[enumerate]{nosep}



\newcommand{\myTitle}{\inColor{\fontsize{1.3cm}{1em}\selectfont yLectureNote}}



\begin{document}
	\everyrow{\tabucline[.4mm  white]{}}
	\taburowcolors[2] 2{tableLineOne .. tableLineTwo}
	\tabulinesep = ^4mm_3mm
	

%%%%%%%%%%%%%%%%%%%%%%%%%%%%%%%%%%%%%%
%		Title
%%%%%%%%%%%%%%%%%%%%%%%%%%%%%%%%%%%%%%

	\begin{flushleft}
		\begin{minipage}{\widthof{\myTitle}}
			{\fontsize{.6cm}{1em}\selectfont\color{mainColor}
				Documentation
			}
			\begin{spacing}{3}
				\myTitle
			\end{spacing}
			\vspace*{-10mm}
			\begin{flushright}
				Yves Zumbach
			\end{flushright}
		\end{minipage}
	\end{flushleft}
	
%	\begin{minipage}[b]{\textwidth}
%		\begin{multicols}{2}
%			{
%				\hypersetup{linkcolor=black}
%				\tableofcontents
%			}
%		\end{multicols}
%	\end{minipage}
	
	\vspace*{2cm}
	
	
	\begin{multicols}{2}
		\section{Class Options}
		\begin{itemize}
			\item french change the document settings to be in French (it is by default in English)
			\item german change the document settings to be in German
			\item printSerie prints the Serie number in the footer if you used the \lstinline|\nextSerie| command
		\end{itemize}
		
		
		\section{Layout}
		Page without margin paragraph are created using \lstinline|\nomarginparpage|. You restore margin paragraph with \lstinline|\marginparpage|.
		
		
		\section{Title Page}
		Use following commands in the preamble to define the title page's content: \lstinline[breaklines]|\title{<class title>} \date{<semester and year>} (\author{<student>}) (\professor{<professor>}) (\yLanguage{<language>})|. The commands in parenthesis are not mandatory.
		
		Inside the document body, issue \lstinline|\titleOne[<additional rubber content>]|.
		
		
		\section{Table of Content}
		\lstinline[breaklines]|\yTableOfContent[<additional content before new page>]|. One could put an authorBlock in the additional content.
		
		\lstinline[breaklines]|\printMarginPartialToc| prints a chapter partial table of content in the margin.
		
		
		\section{Margin Elements}
		\lstinline[breaklines]|\sideNote{<content>}| should be used whenever possible. It adds a note number. You can use \lstinline[breaklines]|\sideTitle{<title>}| inside.
		
		Inside floats or buggy environments: \lstinline[breaklines]|\forcedSideNote{}|. It appears at the exact same height it is declared without checking if it overlaps with other sideNotes.
		
		\lstinline[breaklines]|\marginElement{<content>}| is a simple margin par with no numbers.
		
		
		\section{Theorem, axiom, etc.}
		\begin{lstlisting}[breaklines]
\begin{theorem}[<title>]
	<theorem>
\end{theorem}
		\end{lstlisting}
		
		Following environments are also defined with the same set of arguments: lemma, corollary, definition, axiom, proposition.
		
		
		\section{InfoBulles}
		\verb|\infoInfo{<Title>}{<Text>}|
		
		With the exact same arguments: \verb|\tipsInfo|, \verb|\warningInfo|, \verb|\criticalInfo|,
		\verb|\checkInfo|, \verb|\questionInfo|.
		
		
		\section{Side InfoBulles}
		\lstinline[breaklines]|\sideInfo[<title>]{content}|. With the same arguments: \lstinline[breaklines]|\sideTips \sideCritical|
		
		
		\section{Class Date}
		\lstinline[breaklines]|\classDate{<day>}{<month>}{<year>}|
		
		
		\section{Exercises Set}
		\lstinline[breaklines]|\nextExerciseSet|. Use also this command to \enquote{activate} the first set and make \LaTeX\ display the set number in the footer.
		
		\lstinline[breaklines]|\stopPrintingExerciseSet \printExerciseSet| to stop or reactivate printing of the set number.
		
		
		\section{Lists}
		In the margin par, use \lstinline[breaklines]|sideDescription, sideItemize, sideEnumerate| instead of the normals.
		
		
		\section{Tables}
		Body tables:
		\begin{lstlisting}
(\blockmargin)
\begin{SCtable}[][ht!]
\begin{tabu}{<cols>}
   \tableHeaderStyle
   first & line & of & the table\\
   other & lines & of & the & table\\
\end{tabu}
\caption{<caption text>}
\end{SCtable}
(\unblockmargin)
		\end{lstlisting}
		
		Side Tables: \lstinline[breaklines]|\sideTable[<caption>]{\begin{tabu}...\end{tabu}}|
		
		
		\section{Figures}
		Side figures: \lstinline[breaklines]|\sideFigure[<caption>]{\includegraphics{path.jpg}}|
		
		
		\section{Code Listings}
\begin{lstlisting}
\begin{CodeInfo}{<Title>}[<caption>]
	\begin{CodeInfoLst}[<language>]
		var test = "my Code"
	\end{CodeInfoLst}
\end{CodeInfo}
\end{lstlisting}

		With <language> a string like: \enquote{Python}, \enquote{C++}, etc.
		
		
		\section{Full Width Elements}
		\lstinline[breaklines]|\begin{whole}...\end{whole}|
		
		
		\section{End of Chapter Ornaments}
		\lstinline[breaklines]|\yOrnament|
		
		
		\section{Author Block}
		\begin{lstlisting}[breaklines]
\authorBlock{
	\authorName{<your name>}
	\authorAdressLineOne{<adress line one>}
	\authorAdressLineTwo{<adress line two>}
	\authorAdressLineThree{<adress line three>}
	\authorPhone{<phone number>}
	\authorMail{<mail>}
	\authorWebsite{<website>}
}
		\end{lstlisting}
		
		
		\section{Numbers typesetting}
		\lstinline[breaklines]|\bigNumber{<number>}| Write the number in big in the text. For side numbers: \lstinline[breaklines]|\sideNumber{<number>}{<text>}|
		
		
		\section{Metadata}
		\begin{lstlisting}[breaklines]
\hypersetup{
	pdftitle={<Title>},
	pdfsubject={<Subject>},
	pdfauthor={<Your name>},
	pdfkeywords={{<keyword 1>}{<keyword 2>}},
}
		\end{lstlisting}
		
		
		\section{Quotation}
		\lstinline[breaklines]|\blockQuote[<author>]{<text>} \sideQuote[]{} \fullQuote[]{}|
	\end{multicols}
	
	
\end{document}