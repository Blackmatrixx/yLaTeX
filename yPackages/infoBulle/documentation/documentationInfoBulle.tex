% !TeX spellcheck = en_GB
%%%%%%%%%%%%%%%%%%%%%%%%%%%%%%%%%%%%%%
%		Basic configuration
%%%%%%%%%%%%%%%%%%%%%%%%%%%%%%%%%%%%%%
\documentclass[a4paper, 11pt, oneside, fleqn]{article}
\usepackage[no-math]{fontspec}

\usepackage{polyglossia}
\setdefaultlanguage{french}
\setotherlanguages{english}

\typeout{For commercial use of this Work or a Derived or Compiled Work (as defined in the LPPL v1.3c), contact me before at thib1235[at]gmail.com}



%%%%%%%%%%%%%%%%%%%%%%%%%%%%%%%%%%%%%%
%		Various packages
%%%%%%%%%%%%%%%%%%%%%%%%%%%%%%%%%%%%%%
\usepackage{metalogo} % typeset xelatex!
\usepackage{microtype}
\usepackage{graphicx}
\usepackage{wrapfig}
\usepackage[german=swiss]{csquotes}
\usepackage{calc}
\usepackage[usenames,dvipsnames,svgnames,table]{xcolor}
\usepackage{amsmath, amsfonts, amssymb}
\usepackage{appendix}
\usepackage{setspace}
\usepackage{infoBulle}
\usepackage{lipsum}



%%%%%%%%%%%%%%%%%%%%%%%%%%%%%%%%%%%%%%
%		Colors
%%%%%%%%%%%%%%%%%%%%%%%%%%%%%%%%%%%%%%
%\definecolor{mainColor}{RGB}{150, 150, 150} % a sort of light gray
\definecolor{mainColor}{RGB}{211, 47, 47} % some dark red

\newcommand{\inColor}[1]{{\bfseries\color{mainColor}#1}}



%%%%%%%%%%%%%%%%%%%%%%%%%%%%%%%%%%%%%%
%		Font
%%%%%%%%%%%%%%%%%%%%%%%%%%%%%%%%%%%%%%
\defaultfontfeatures{Ligatures=TeX}
\frenchspacing
% For source code
\setmonofont{Source Code Pro Light}[
BoldFont=Source Code Pro,
]
% Normal font
\setsansfont{Lato Light}[
Numbers=OldStyle,
BoldFont=Lato Regular,
ItalicFont=Lato Light Italic,
BoldItalicFont=Lato Italic
]
% Normal font
\setmainfont{Lato Light}[
Numbers=OldStyle,
BoldFont=Lato Regular,
ItalicFont=Lato Light Italic,
BoldItalicFont=Lato Italic
]
% Font for section, subsection, subsubsection, etc
\newfontfamily{\titlefont}{Lato Light}[
Numbers=OldStyle,
BoldFont=Lato Regular,
ItalicFont=Lato Light Italic,
BoldItalicFont=Lato Italic
]
% Font for chapter number
\newfontfamily{\upperNumber}{Lato Light}[
BoldFont=Lato Regular,
ItalicFont=Lato Light Italic,
BoldItalicFont=Lato Italic
]

\newfontfamily{\FA}{fontawesome-webfont.ttf}



%%%%%%%%%%%%%%%%%%%%%%%%%%%%%%%%%%%%%%
%		Layout
%%%%%%%%%%%%%%%%%%%%%%%%%%%%%%%%%%%%%%
\usepackage[
xetex,
a4paper,
ignoreheadfoot,
left=3.5cm,
right=3.5cm,
top=3.5cm,
bottom=3.5cm,
nohead,
marginparwidth=0cm,
marginparsep=0mm
]{geometry}
\setlength{\skip\footins}{1cm}
\setlength{\footnotesep}{2mm}
\setlength{\parskip}{1ex}
\setlength{\parindent}{0ex}



%%%%%%%%%%%%%%%%%%%%%%%%%%%%%%%%%%%%%%
%		Titling
%%%%%%%%%%%%%%%%%%%%%%%%%%%%%%%%%%%%%%
\usepackage{titlesec}
% Tiltling format (font size)
\titleformat*{\section}{\Huge\titlefont}
\titleformat*{\subsection}{\huge\titlefont}
\titleformat*{\subsubsection}{\LARGE\titlefont}

%Titling spacing
\titlespacing*{\section}{0mm}{4mm}{3mm}
\titlespacing*{\subsection}{0mm}{3mm}{2mm}
\titlespacing*{\subsubsection}{0mm}{2mm}{1.5mm}

% Title number in margin
\makeatletter
\def\@seccntformat#1{\llap{\csname the#1\endcsname\hspace{3mm}}}
\makeatother



%%%%%%%%%%%%%%%%%%%%%%%%%%%%%%%%%%%%%%
%		Tables
%%%%%%%%%%%%%%%%%%%%%%%%%%%%%%%%%%%%%%
\usepackage{array}
\usepackage{tabu}
\usepackage{longtable}

\definecolor{tableLineOne}{RGB}{245, 245, 245}
\definecolor{tableLineTwo}{RGB}{224, 224, 224}



%%%%%%%%%%%%%%%%%%%%%%%%%%%%%%%%%%%%%%
%		Links
%%%%%%%%%%%%%%%%%%%%%%%%%%%%%%%%%%%%%%
\usepackage{hyperref}
\hypersetup{
	pdftoolbar=false,
	pdfmenubar=true,
	pdffitwindow=false,
	pdfborder={1 1 0},
	pdftitle={Title},
	pdfauthor={Your name},
	pdfsubject={},
	pdfcreator=LaTeX,
	pdfkeywords={{keyword 1}{keyword 2}},
	colorlinks=true,
	linkcolor=blue,
	linktoc=all,
	urlcolor=blue,
	citecolor=blue,
	filecolor=blue
}



%%%%%%%%%%%%%%%%%%%%%%%%%%%%%%%%%%%%%%
%		Bibliography
%%%%%%%%%%%%%%%%%%%%%%%%%%%%%%%%%%%%%%
%\usepackage[
%	sorting=none,
%	defernumbers=true,
%	isbn=false,
%	backend=biber
%]{biblatex}
%\addbibresource{./biblio.bib}



%%%%%%%%%%%%%%%%%%%%%%%%%%%%%%%%%%%%%%
%		Itemize and consort
%%%%%%%%%%%%%%%%%%%%%%%%%%%%%%%%%%%%%%
\def\labelitemi{---}
\usepackage{enumitem}
\setlist[itemize]{nosep}
\setlist[description]{nosep}
\setlist[enumerate]{nosep}



\newcommand{\myTitle}{\inColor{\fontsize{1.3cm}{1em}\selectfont InfoBulle Package}}



\usepackage{infoBulle}

\begin{document}


\everyrow{\tabucline[.4mm  white]{}}
\taburowcolors[2] 2{tableLineOne .. tableLineTwo}
\tabulinesep = ^4mm_3mm


\hfill
\begin{minipage}{\widthof{\myTitle}}
	{\fontsize{.6cm}{1em}\selectfont\color{mainColor}
		Documentation
	}
	\begin{spacing}{3}
		\myTitle
	\end{spacing}
	\vspace*{-10mm}
	\begin{flushright}
		Yves Zumbach, january 2016
	\end{flushright}
\end{minipage}

\newpage

\section{Preamble}
This package is freely available under the \LaTeX\ Project Public Licence. For commercial use of this Work or a Derived or Compiled Work (as defined in the LPPL v1.3c), contact me before at thib1235[at]gmail.com

\section{Include the package}
\lstinline|\usepackage{infoBulle}|

\section{Prerequisites}
To use this package, you need to compile the document using it with \inColor{\XeLaTeX}. You also need the \inColor{fontawesome} package and font installed on your system.

If you can't fit those requirements, use the previous versions of this package. However, note that lots of functionnalities were added thanks to the fontawesome package (which requires \XeLaTeX).


\section{Defined commands}
\subsection{infoBulles}
This package mainly defines the following commands:
\begin{itemize}
	\item \verb|\infoInfo{<Title>}{<Text>}|
	\item \verb|\warningInfo{<Title>}{<Text>}|
	\item \verb|\criticalInfo{<Title>}{<Text>}|
	\item \verb|\checkInfo{<Title>}{<Text>}|
	\item \verb|\questionInfo{<Title>}{<Text>}|
\end{itemize}

Which results in this:
\infoInfo{Title}{\lipsum[3]}
\warningInfo{Title}{\lipsum[3]}
\criticalInfo{Title}{\lipsum[3]}
\checkInfo{Title}{\lipsum[3]}
\questionInfo{Title}{\lipsum[3]}

\subsection{Code listings}
This package also implement an environment that typeset code listings:
\begin{lstlisting}
\begin{CodeInfo}{<Title>}[<caption>]
	\begin{CodeInfoLst}[<language>]
var test = "my Code"
	\end{CodeInfoLst}
\end{CodeInfo}
\end{lstlisting}

The caption and the language arguments are optional. \verb|<language>| is the language in which the listed code is written, as given in the list of supported languages in the listings package documentation.

Which results in this:


\begin{CodeInfo}{Titre}
	\begin{CodeInfoLst}
var test = "my Code"
	\end{CodeInfoLst}
\end{CodeInfo}

\section{Customization}
It is possible to define new infoBulle (here levelInfo) using the following syntax (before \verb|\begin{document}|):

\begin{lstlisting}
\makeatletter

\definecolor{levelColor}{RGB}{23, 23, 23}

\newcommand{\levelInfo}[2]
{\@infoBulleSkeleton{levelColor}{<logo>}{#1}{#2}}

\makeaatother
\end{lstlisting}

The \verb|\definecolor| command defines a new color that will be used for the line and logo color (passed as first argument to \verb|\@infoBulleSkeleton}|). The \verb|<logo>| argument should be a fontawesome command or a tikzpicture ($0.75cm \times 0.75cm$).

Example:
\begin{lstlisting}
% Just pass a fontawesome icon
\faTimesCircle
\end{lstlisting}

\begin{lstlisting}
% Or a complete tikzpicture
\newcommand{\codeLogo}{
	\begin{tikzpicture}[inner sep=0mm] %remove white margin
		\fill[Grey] (0,0) circle (.35cm); % logo background
		\draw (0,0)
			node {\fontsize{.4cm}{.48cm}\color{White}A};
	\end{tikzpicture}
}
\end{lstlisting}

\end{document}


