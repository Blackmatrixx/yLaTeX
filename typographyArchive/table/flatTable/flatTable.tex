\documentclass[a4paper, 10pt, oneside, fleqn]{report}
\usepackage[no-math]{fontspec}

\typeout{For commercial use of this Work or a Derived or Compiled Work (as defined in the LPPL v1.3c), contact me before at thib1235[at]gmail.com}

\usepackage{polyglossia}
\setdefaultlanguage{french}
\setotherlanguages{english}
\usepackage{calc}
\usepackage[usenames,dvipsnames,svgnames,table]{xcolor}
\usepackage{microtype}
\usepackage{csquotes}



%%%%%%%%%%%%%%%%%%%%%%%%%%%%%%%%%%%%%%
%		Layout
%%%%%%%%%%%%%%%%%%%%%%%%%%%%%%%%%%%%%%
\usepackage[
xetex,
%showframe,
a4paper,
ignoreheadfoot,
left=3.5cm,
right=2.7cm,
top=3cm,
bottom=3.5cm,
nohead,
marginparwidth=0cm,
marginparsep=0mm
]{geometry}
\setlength{\skip\footins}{1cm}
\setlength{\footnotesep}{2mm}
\setlength{\parskip}{1ex}
\setlength{\parindent}{0ex}



%%%%%%%%%%%%%%%%%%%%%%%%%%%%%%%%%%%%%%
%		Colors
%%%%%%%%%%%%%%%%%%%%%%%%%%%%%%%%%%%%%%
\definecolor{lightGray}{RGB}{150, 150, 150}



%%%%%%%%%%%%%%%%%%%%%%%%%%%%%%%%%%%%%%
%		Font
%%%%%%%%%%%%%%%%%%%%%%%%%%%%%%%%%%%%%%
\defaultfontfeatures{Ligatures=TeX}
\frenchspacing
% Normal font
\setsansfont{Lato Light}[
Numbers=OldStyle,
BoldFont=Lato Regular,
ItalicFont=Lato Light Italic,
BoldItalicFont=Lato Italic
]
% Normal font
\setmainfont{Lato Light}[
Numbers=OldStyle,
BoldFont=Lato Regular,
ItalicFont=Lato Light Italic,
BoldItalicFont=Lato Italic
]



%%%%%%%%%%%%%%%%%%%%%%%%%%%%%%%%%%%%%%
%		Tables
%%%%%%%%%%%%%%%%%%%%%%%%%%%%%%%%%%%%%%
\usepackage{array}
\usepackage{tabu}
\usepackage{longtable}

\definecolor{tableHeader}{RGB}{211, 47, 0}
\definecolor{tableLineOne}{RGB}{245, 245, 245}
\definecolor{tableLineTwo}{RGB}{224, 224, 224}





\begin{document}
	
	\everyrow{\tabucline[.4mm  white]{}}
	\taburowcolors[2] 2{tableLineOne .. tableLineTwo}
	\tabulinesep = ^4mm_3mm
	
	
	\begin{table}[ht!]
		\begin{tabu} to \textwidth {l X X[1.2] X[1.2] l }
			\rowfont{\bfseries\sffamily\leavevmode\color{white}}
			\rowcolor{tableHeader}
			& Ressource & Comment en gagner & Comment en perdre & \\
			& Argent & Compléter des quêtes,\newline Dans les coffres & Achats & \\
			& Énergie verte & Planter une graine,\newline Se régénère avec le temps & Utiliser la violence,\newline Lancer un sort & \\
			& Coefficient de naturalité & Faire preuve de naturalité,\newline Compléter des sous-quêtes & Utiliser la violence & \\
		\end{tabu}
		\caption{La gestion des ressources}
	\end{table}
	
	\begin{longtabu} to \textwidth {>{\bfseries}X[r, 1] X[4]}
		\rowfont{\bfseries\sffamily\leavevmode\color{white}}
		\rowcolor{tableHeader}
		Mot & Définition\\
		\endhead
		\endfoot
		%
		Add-on & Un add-on, ou plug-in, est un morceau de code qui est ajouté à un programme afin de lui ajouter une ou plusieurs fonctionnalités spécifiques qui ne sont pas fournies par défaut.\\
		%
		Boss & Anglicisme qui désigne l'ennemi principal du jeu ou du niveau. Il peut y avoir plusieurs boss s'ils sont propres aux niveaux. Ces personnages sont, en général, rencontrés en combat au moins une fois. L'exemple le plus connu est \enquote{Bowser}, un dragon diabolique, qui enlève la belle \enquote{Peach} dans le jeu \enquote{Mario}\\
		%
		Démo & Abbréviation de \enquote{version de démonstration}; désigne une version incomplète d'un jeu vidéo. Il peut s'agir d'un nombre diminué de niveaux, de possibilités de jeu bridées ou de toute autre méthode réduisant la version intégrale du jeu.\\
		%
		Gameplay & Anglicisme désignant la façon dont le joueur peu interagir avec le jeu, les actions qui lui sont permises, les objets qui sont à sa disposition, les façons de faire permettant de surmonter les défis du jeu, etc.\\
		%
		HUD & Acronyme de \textit{Head-Up Display}. C'est toute l'interface 2D présente sur l'écran du joueur alors qu'il est dans le jeu (vraiment en train de jouer, pas les menus). Cela peut comprendre par exemple la barre de vie, l'argent disponible, les armes équipées ou encore la minimap. Ce sont des informations qui apparaissent comme s'il y avait une vitre devant vos yeux sur laquelle on projetait des images. Initialement, cela désignait bien des vitres sur lesquelles étaient projetées des informations de vol... dans les cockpit des avions de chasse.\\
		%
		Moteur de jeu & Un moteur de jeu est un programme chargé de centraliser plusieurs aspects de la création de jeu vidéo: importation de contenu (3D ou 2D), calculs pour le rendu des images (dans le cas d'un univers en 3D: projection de cet environnement sur un espace 2D -- l'écran), gestion du son, programmation des actions ou des personnages (ou scripting), exportation vers diverses plateformes, etc.\\
		%
		Open-world & Anglicisme; littéralement \enquote{monde ouvert}. Désigne un jeu vidéo dans lequel le joueur peut se déplacer librement à travers un univers virtuel très vaste. Il y dispose d'une grande liberté d'action.\\
		%
		PNJ & Abréviation tirée du domaine du jeu de rôle, signifiant Personnage Non Joueur, soit n'importe quel personnage qui n'est pas contrôlé par un homme. Ils désignent, dans un jeu vidéo, les protagonistes de l'histoire contrôlés par l'ordinateur.\\
		%
		Puzzle & Une énigme, une devinette ou un mini-jeux à résoudre pour pouvoir avancer dans le jeu, débloquer des trésors cachés ou encore compléter des objectifs secondaires. Ils peuvent être de formes très diverses: épreuve de rapidité, d'adresse, de réflexion, etc. Par exemple, dans le jeu \textit{The Legend of Zelda: Spirit Tracks} il faut reproduire un morceau de musique avec un instrument débloqué au fil de l'aventure pour accéder à certains lieux.\\
		%
		Screenshot & Anglicisme signifiant \enquote{capture d'écran}; autrement dit, une photo de ce qui s'affiche à l'écran.\\
		%
		Semi open-world & De l'anglais, \enquote{monde semi-ouvert}. C'est un niveau ou type de jeu dans lequel le joueur est libre de ses mouvements. Cependant des éléments du décor limitent la taille de l'univers. On trouvera, par exemple, des îles (la mer est l'élément bloquant), des villes fortifiées (les murailles sont les éléments bloquants) ou encore une maison dont on ne peut sortir. Les niveaux de type alley désigne exactement la même chose mais semi open-world ou open-world sont des termes que l'on peut entendre en français. Voir également la définition de \enquote{Open-world}\\
		%
		Succès & Traduction de achievement, ce sont des récompenses honorifiques attribuées au joueur à l'accomplissement d'un tâche donnée, d'objectifs supplémentaires, etc.\\
		%
		Story-telling & Anglicisme; décrit l'art de raconter une histoire efficacement. Cela inclut par exemple: le point de vue narratif, l'intrigue, les personnages mais aussi l'intonation, les gestes, etc. Dans le cadre d'un jeu vidéo, cela décrira les techniques narratives, les libertés laissées au joueur, etc.\\
		%
		Temple & Un temple fait référence, dans l'univers du jeu vidéo, au dernier niveau d'un chapitre ou acte du jeu. Il se déroule généralement dans l'entre de l'ennemi principal ou tout du moins d'un adversaire de grande importance. Ce type de niveau se conclu généralement par un affrontement entre le joueur et l'ennemi cité avant -- appelé alors boss. On citera ainsi Super Mario Bros dans lequel chaque dernier niveau se déroule dans un château de Bowser ou encore Zelda où chaque temple conclu une partie du jeu et coïncide généralement avec l'obtention d'un nouvel objet.\\
		%
		Workflow & Anglicisme, que l'on pourrait traduire par \enquote{flux de travail}. Cela désigne l'ensemble des étapes que vont subir des informations, des documents ou des produits pour passer d'un premier état à un deuxième.\\
	\end{longtabu}
	
\end{document}




