%%%%%%%%%%%%%%%%%%%%%%%%%%%%%%%%%%%%%%
%		Basic configuration
%%%%%%%%%%%%%%%%%%%%%%%%%%%%%%%%%%%%%%
\documentclass{parch}
\usepackage{enumitem}
\usepackage{verbatim}


\begin{document}
	
	\thispagestyle{empty}
	\vspace*{-1cm}{\centering\includegraphics[width=\textwidth]{../../images/logoParchEpure2.png}}
	
	\vspace*{-.2cm}\noindent{\color{mediumGrey}\rule{0.87\textwidth}{1mm}}\hspace{3pt}{décembre 2015}
	
	\vspace*{3\baselineskip}
	\noindent\begin{minipage}{.32\textwidth}
		\begin{parchAbstractTheme}{Sexualité à trous}
			Quelle est la différence entre érotisme et pornographie ? La limite n'est pas clairement définie : dans un texte, par exemple, le critère de vocabulaire peut déterminer l'appartenance à l'un ou l'autre des genres. Mais un mot peut choquer une personne sans rebuter une autre : les goûts varient autant que les personnalités. C'est pourquoi je vous propose un texte, à la base légèrement érotique de mon point de vue, dont une partie est laissée à votre guise ! A vous de choisir comment se déroule la scène, si elle est crue ou cuite, sensée ou absurde !\\\null\hfill\textbf{p. 4}
		\end{parchAbstractTheme}
		
		\begin{parchAbstractTheme}{Feuilleton}
			Une horde de mannequins immaculés tente d'investir le collège, massacrant silencieusement ses habitants terrorisés. Mais quand finira cette affreuse mascarade ? Les collégiens ne verront-ils plus jamais la lumière? Le voile ténébreux couvrira-t-il toujours le monde de son ombre impérieuse ?\\\null\hfill\textbf{p. 3}
		\end{parchAbstractTheme}
		
%		\begin{parchAbstractTheme}{Interview}
%			Une femme… professeur de français depuis 5 ans… bavarde… fan de série TV… \enquote{adorable}… voici l’interview de… \textsc{Madame Baumgartner}!!!\\\null\hfill\textbf{p. 2}
%		\end{parchAbstractTheme}
		
		\begin{parchAbstractTheme}{Typographie}
			Deux techniques pour mettre des éléments de votre texte en évidence tout en conservant une mise en page professionnelle.\\\null\hfill\textbf{p. 3}
		\end{parchAbstractTheme}
	\end{minipage}\hspace*{.03\textwidth}
	\noindent\begin{minipage}{.65\textwidth}
		\noindent\includegraphics[width=\linewidth]{images/couvertureVerticale.png}
		
		\vspace*{2\baselineskip}
		\parchTitreArticle{L'équipe du Parch'\\vous souhaite{\color{White}p}\\bonne chance...}
	\end{minipage}
	
%	\begin{multicols}{3}
%		
%		\vspace*{.7cm}
%		\begin{parchAbstractTheme}{Sexualité à trous}
%			Quelle est la différence entre érotisme et pornographie ? La limite n'est pas clairement définie : dans un texte, par exemple, le critère de vocabulaire peut déterminer l'appartenance à l'un ou l'autre des genres. Mais un mot peut choquer une personne sans rebuter une autre : les goûts varient autant que les personnalités. C'est pourquoi je vous propose un texte, à la base légèrement érotique de mon point de vue, dont une partie est laissée à votre guise ! A vous de choisir comment se déroule la scène, si elle est crue ou cuite, sensée ou absurde !\\\null\hfill p. 4
%		\end{parchAbstractTheme}
%		
%		\vspace*{.7cm}
%		\begin{parchAbstractTheme}{Feuilleton}
%			Une horde de mannequins immaculés tente d'investir le collège, massacrant silencieusement ses habitants terrorisés. Mais quand finira cette affreuse mascarade ? Les collégiens ne verront-ils plus jamais la lumière? Le voile ténébreux couvrira-t-il toujours le monde de son ombre impérieuse ?\\\null\hfill p. 3
%		\end{parchAbstractTheme}
%		
%		\noindent\includegraphics[width=2\linewidth]{images/couvertureVerticale.png}
%		
%		\vspace*{9.5mm}
%		\begin{parchAbstractTheme}{Interview}
%			Une femme… professeur de français depuis 5 ans… bavarde… fan de série TV… \enquote{adorable}… voici l’interview de… \textsc{Madame Baumgartner}!!!\\\null\hfill p. 2
%		\end{parchAbstractTheme}
%		
%		\null
%		\vspace*{12.5cm}
%		\begin{parchAbstractTheme}{Typographie}
%			Deux techniques pour mettre des éléments de votre texte en évidence tout en conservant une mise en page professionnelle.\\\null\hfill p. 3
%		\end{parchAbstractTheme}
%	\end{multicols}
	
	\newpage
	
	\noindent\parchArticleCoteACote{
		\parchTitreArticle{Édito}
		
			\vspace*{\baselineskip}Avec la COP21, on entend tous parler de l'environnement. C'est un sujet bien en vogue ces temps. Mais que pouvons-nous faire réellement, en tant qu'individu ?
			
			Eh bien, nous pouvons repenser notre quotidien car chacune de nos actions a un impact. Je pense ici plus particulièrement au coût énergétique d'Internet et de tout ce qui est lié au web. Chaque recherche sur Google pollue ! En fait, internet représente 8\% de la consommation électrique suisse. L'apparente immatérialité d'Internet n'empêche pas les infrastructures nécessaires au maintien de ce réseau de polluer fortement. Autre exemple, utiliser son téléphone dix minutes par jour pendant un an revient à parcourir 80 km en voiture (c'est plus qu'il n'en faut pour aller jusqu'à Lausanne). Imaginez les dégâts avec une heure de téléphone par jour... Il devient crucial de comprendre les implications de nos actes afin de pouvoir mieux les maîtriser et limiter notre impact.
			
			Que faire alors ? Eh bien, il existe des trucs simples pour éviter de polluer trop. Ne pas changer de téléphone tous les ans et ne pas changer d'ordinateur tous les trois ans permet d'éviter le gaspillage de ressources précieuses. Vous pouvez aussi rentrer l'url des sites web directement dans la barre d'adresse plutôt que de passer par la recherche Google. Finalement, limitez le nombre d'e-mails que vous envoyez, évitez les pièces jointes volumineuses, envoyez toutes les informations en un message, etc. Bien sûr, ces principes peuvent s'appliquer de la même façon à What's App…
			
			Sur ce, je vous laisse; je vais reprendre mon vélo en espérant que le réchauffement climatique arrive suffisamment rapidement pour que je n'aie plus froid en pédalant.
			
			\parchAuteurLieu{Yves Zumbach}{Dans le froid}
	}
	{
		\begin{parchArticle}[2]{L’adorable bavarde}
			{Chiara et Valentina}{Salle 232}{1}
			\begingroup
			\setlength{\parindent}{0pt}
			
			\parchImage{images/mmeBaumgartner.png}
			
			\textbf{-- Pouvez-vous vous présenter en quelques mots?}
			
			Je m’appelle Iris Baumgartner… j’enseigne le français au collège Madame de Staël.
			
			\textbf{-- Quel a été votre parcours scolaire et professionnel?}
			
			J’ai fait ma scolarité dans le canton de Vaud, donc j’ai fini, du coup, ma maturité une année plus tôt parce qu’on a que trois ans de collège dans le canton de Vaud ce qui est assez sympa. Et puis ensuite j’ai pris une année sabbatique et… euh… je suis partie à Zurich faire des études de traduction donc j’étais là-bas quelques années. J’ai bossé là-bas une année et en suite j’suis partie voyager assez longtemps et j’suis revenue… euh… ensuite faire… euh… faire des études… euh… de lettres à l’université de Genève. Et après quelques années… euh… j’ai un peu hésité entre la traduction et puis, puis… euh… peut-être l’enseignement. J’ai commencé à travailler ici en 2010.
			
			\textbf{-- Quel genre d’élève étiez-vous?}
			
			… (silence)… Bavarde (rire x2), c’est pas une surprise je pense… euh… et puis plutôt timide.
			
			\textbf{-- Quelles étaient vos branches préférées?}\newpage
			
			J’aimais bien les langues en fait, allemand, anglais surtout, le français aussi mais un peu moins (… ??...) et l’histoire. J’aimais pas les sciences et pourtant j’étais en scientifique.
			
			\parchStampRight{Interview}\textbf{-- Vous vous considérez plutôt gentille, méchante, sévère ou juste (en tant que professeur)?}
			
			\enquote{Adorable}, bien sûr (rires). Très très très gentille (rire), je sais pas, moi j’ai pas l’impression d’être particulièrement sévère.
			
			\textbf{-- Quelle a été votre plus grande honte?}
			
			Une situation que je trouve toujours très très désagréable c’est d’avoir dit du mal de quelqu’un qui se situait juste à côté de moi. Ça, ça je dois dire c’était très très embarrassant.
			
			\textbf{-- Quelles sont vos hobbies?}
			
			Le sport, la fiction et les séries TV.
			
			\textbf{-- Pires défauts?}
			
			J’suis \enquote{bavarde}… j’suis aussi, parfois, un peu perfectionniste, ce qui peut me rendre parfois un peu rigide sur certaines choses (rires), et puis j’sais pas après si j’ai des défauts extrêmement voilà… [censuré]…
			
			\textbf{-- Quelles sont vos plus grandes qualités?}
			
			Je suis capable de faire preuve d’auto-dérision, je dirais que c’est une bonne qualité quand on est prof, surtout qu’on se met parfois dans des situations un peu embarrassantes... (rires). Comme maintenant par exemple (rires) donc de pouvoir en rire, c’est quand même une qualité,  je dirais qui est assez importante.
			
			\textbf{-- Quel est votre sport préféré?}
			
			Préféré préféré préféré… je dirais la natation.
			\endgroup
		\end{parchArticle}
	}
	
	\newpage
	
	
	\begin{multicols}{3}
		\begin{spacing}{1}
		\begin{gridenv}
			\parchTitreArticle{Cinétique}
		\end{gridenv}
		\vspace*{1.5mm}
		
		Un monstre s'avance – la flèche libérée quelques grains de temps plus tôt par Terne le fixe comme un regard de Gorgone, le laissant, là, droit comme une croix.
		
		\parchImage{images/illustrationFeuilletonCoupee.png}
		
		Un autre, sur un bureau perché, s'élance --- et reçoit le trait tracé par Tranj, dont la main vibre encore. Terne, en train de recharger, pense : \enquote{Ils sont quand même beaucoup.}
		
		Un troisième écarte de son chemin les deux précédentes\parchStampLeft{Roman-Feuilleton} engeances --- Neliot ajuste, le touche.
		
		Mais juste derrière surgit un monstre de trop qui, bras avides, se jette sur lui et le renverse; Tranj s'interpose --- l'aberration le balaie d'un mouvement inhumain. Elle se dirige vers Terne, lui arrache son arc, la pousse violemment contre le mur; commence à se pencher doucement de toute son horreur, excessive, effrayante, écrasante; approche précautionneusement ses mains minces et son visage vide --- bondit en arrière, désarticulée, en subissant l'impulsion centrifuge d'une massue. Sa propriétaire ramène lestement l'instrument (c'est un de ces pieds à pendule qu'on peut trouver dans les salles de physique), esquissant à la volée un sourire féroce. \enquote{Il faut qu'on se replie !} lance Tranj, se relevant. Derrière lui, quelques élèves se munissent des autres arcs. Mais la masse roulante des cracheurs de gaz, qui vient d'engloutir les derniers valeureux guerriers du grand hall, en déborde déjà. Les collégiens se précipitent vers les marches et franchissent acrobatiquement une muraille de bureaux établie au premier étage. Là, ils sont une centaine à empiler toujours plus de meubles. Quelqu'un crie : \enquote{On va leur jeter des chaises dessus !}
		
		Le flot furieux de la marée de monstres remplit soudain la cage d'escalier et assiège, bouillonnant d'écume rageuse, les ruines branlantes d'un passé idéalisé, dont il ne subsiste qu'une impression. Mais déjà les chaises volent gracieusement, chutent, coiffent ces chauves têtes de quille. Les flèches se noient sans bruit dans le mélange des corps secs. Le liquide lourd des anges blancs se gazéifie, se fonce, fume: chaque coup fait fuir du brouillard de leurs entrailles. Leurs assauts se dégonflent, la digue reste debout; la tempête se calme, la marée baisse, les vagues meurent. Seule une petite brise reste.
		
		Les guerriers --- professeurs et élèves --- revinrent de leur folie hargneuse comme d'un rêve. Tandis que les bras tombaient,\parchStampRight{Roman-Feuilleton} les c\oe urs se calmaient et les genoux pliaient, un silence profond s'installa, plein d'un recueillement sacré. L'atmosphère se détendit, se troublant parfois de sanglots choqués ou de soupirs fatigués.
		
		Terne se présenta à la fille à la massue. \enquote{Je m'appelle Bessy, répondit-elle.\\
			-- Merci pour le couloir. J'ai cru arriver à la fin de ma vie...\\
			-- On y a tous cru. Mais on s'est pas laissé faire ! En tout cas, c'est fini maintenant.\\
			-- Bah non. On est toujours dans le noir, coupés du monde, lança Neliot, las.\\
			-- Si les monstres sont partis, on peut aller voir de quoi est fait ce mur!} La proposition de Tranj plut. Franchissant le rempart, ils foulèrent les marches recouvertes de cendres laissées par les cadavres blancs qui se décomposaient en fumant. Un courant d'air siffla, dispersant la poussière. Neliot s'approcha de la fenêtre et scruta la nuit: il lui sembla distinguer la limite du voile sombre. Une bourrasque fraîche parcourut l'intérieur du collège, faisant frissonner ses habitants.
		
		Et la lumière fut.
		
		\begin{gridenv}
			\parchAuteurLieu{Yann}{Dans sa bulle}
		\end{gridenv}
		
		\parchHRule
		
		\begin{gridenv}
			\parchTitreArticle{Mise en\\évidence}
		\end{gridenv}
		
		\vspace*{-.5mm}
		\noindent Il existe deux manières de mettre du texte en évidence. La première consiste à utilise le gras. Cette mise en évidence est très forte : si vous tenez la page à un mètre de vous, vous remarquez tout de suite cet exergue. C'est la méthode la plus communément utilisée dans les textes modernes. L'autrer méthode consiste à utiliser l'italique. Elle n'est visible qu'à la lecture du texte. C'est une mise en exergue beaucoup plus subtile, utilisée surtout dans les mises en page plus classiques.
		
		Dans tous les cas, le souligné ne s'utilise jamais ! Et bien qu'il soit possible d'imaginer mettre de la couleur pour les emphases, je déconseille fortement cette méthode ; elle ne se justifie que dans des cas très particuliers et participe souvent à rendre le texte moins lisible.
		
		Dans vos documents, n'utilisez qu'un seul des deux types de mise\parchStampRight[0]{Bref'Typo} en exergue. Définissez à l'avance si le lecteur doit pouvoir voir immédiatement de quoi il retourne sans forcément lire le texte (gras) ou si le lecteur doit lire votre texte pour le comprendre (italique).
		
		\parchAuteurLieu{Yves Zumbach}{Jouant avec de la fonte}
		
			\end{spacing}
		\end{multicols}
	
	\newpage
	
	\begin{parchArticle}{Vieillesse ennemie}
		{Garance Sallin}{Dans le hall}{1}
		\begin{parchResumeTheme}{Prosopopée}
			\enquote{Venez voir mourir le dernier sex-symbol, venez tous applaudir à la fin d’une idole}; c’est une âme oubliée qui prend la parole ce mois-ci…
		\end{parchResumeTheme}
		
		\parchImage{images/bruitsDeCouloir.jpg}
		
		Le cours est suspendu.
		
		Et je voyais vos visages s’illuminer. La surprise, la joie, la réjouissance. Vos camarades moins chanceux vous traitaient de tous les noms. Parfois, il y avait des nouvelles\parchStampLeft{Bruits de couloir} qui vous concernaient mais qui vous exaspéraient malgré tout. Dans un long bâillement, vous retourniez chez vous pour tenter de rattraper ces deux heures de sommeil…
		
		Le bonheur et l’envie de transmettre anime les artistes comme les enseignants. Je n’étais ni l’un, ni l’autre. J’étais une simple feuille de papier bien protégée par mon armure de verre. Vos yeux plissés se reflétaient sur la vitrine, j’étais aussi convoitée qu’une robe de haute couture ou que les anciens sex-symbols. Mais voilà une époque révolue, et ma vie n’est plus qu’oubli et pénombre.
		
		C’est une jeunette qui a pris ma place. Toute pimpante, toute brillante, toute moderne. Elle vous affuble de couleurs chatoyantes et vous ravit\parchStampRight{Bruits de couloir} les yeux dans son élégant habit noir ; elle vous a déjà tous conquis. Qui se souvient de moi ? Qui se rappelle le temps où je m’étendais sur ce lit de liège et où c’était moi qui vous rendais heureux ? Le temps passe et les idoles se remplacent. Les pixels succèdent à l’encre, l’écran à la feuille. Et je meurs.
		
		Le cours est suspendu…
	\end{parchArticle}
	
	\begin{parchArticle}{Discontinuité suggestive}
		{Yann}{...}{1}
		\begin{parchResumeTheme}{... (à remplir!)}
			... (devine!)
		\end{parchResumeTheme}
		Enfoncée dans le fauteuil moelleux, elle semble vouloir s'en extirper. Des petites bottines lacées jaillissent ses longues jambes que les bas enserrent et assombrissent, leur donnant un teint mat et accrocheur. L'orée d'une jupe courte colle, tendue, ses cuisses musclées. Cette tension remonte le long des hanches, dessinant une courbe harmonieuse qui vient se reposer sur les reins, où la jupe laisse filer un haut …………. (adjectif) enveloppant subtilement son buste et ses bras. Son cou fin a pour aboutissement un visage magnifique, aux yeux à ailes amples et lèvres à la moue moqueuse. Ses cheveux chatoyants sont retenus par un chignon. Elle le fixe, lui qui la regarde sans rien dire : elle a l'air de s'impatienter.
		
		Étendu sur son trône, il l'observe d'un regard provocateur. Sa chevelure rebelle couronne des traits angéliques : lèvres pulpeuses et yeux clairs, que sa barbe foncée --- crépitement d'arabesques soyeuses – fait briller. La musculature développée de ses épaules gondole l'épaisse encolure de son grand manteau …………. (adjectif) ouvert ; sa respiration profonde fait onduler son torse puissant, couvert d'un doux duvet de cachemire. Ses jambes, nonchalantes dans leur étui de toile beige, se concluent élégamment par des mocassins de daim.
		
		Enfin, elle se lève, s'approche silencieusement de lui, pose la main sur son …………., le caresse puis remonte, épouse des doigts les formes de son …………. Il l'attire à lui en la prenant par la …………., la ramène contre son …………., promène son autre main sur sa …………. Ils s'embrassent langoureusement. Elle fait mine de se retirer, il resserre son étreinte ; elle le repousse si violemment qu'il chute avec son fauteuil. Il se relève lestement en roulant en arrière, voit son air amusé, prend une expression faussement outrée et jette théâtralement son manteau. Il s'avance, fier, tandis qu'elle défait son chignon. Alors qu'ils se mesurent, muets, au milieu de la pièce, il bascule la tête vers sa …………., l'explore de sa …………. ; elle applique d'abord doucement les mains sur son …………. puis le masse délicieusement fort. Soudain, il veut se venger et la pousse vers le lit, mais elle l'entraine avec : ils tombent et les lattes lâchent toutes au même moment. Essayant avec peine de ne pas rire, ils commencent à se déshabiller dans une lutte qui achève l'armature défoncée du lit. Elle lui ôte s.. …………., il lui vole s.. …………. ; finalement, ils se retrouvent à demi nus. Lui est dessus/dessous : il la …………. (verbe) avec s.. ………….. ; elle positionne sa …………. ………. (nom + préposition) son …………. ; ils roulent et, en se débattant, elle projette sa bottine droite/gauche sur la photo d.. ………….. (nom), trainant sur une commode, qui explose sous le choc. Les positions sont inversées. La rixe se prolonge en tripatouillages et rires retenus, jusqu'à ce qu'ils tombent du matelas. Ils se relèvent, se tournent autour, puis se reprennent par une danse effrénée qui les fait voltiger dans les rideaux pourpres ; un craquement, et le tissu léger leur tombe sur les épaules, les entrave, et ils culbutent au sol. Alors qu'enfin ils se confondent, cachés dans cette couverture opaque, on entend un « …………. (chant du plaisir extrême)» étouffé. Brusquement, l'autre s'arrête, triomphe : « Ha ! Perdu ! » et administre une tapette sur sa………….
	\end{parchArticle}
	
	\begin{parchArticle}{Les 10 commandements\\pour réussir ses trimes}
		{Une petite quatrième}{Libérée par la grève}{1}
		\begin{parchResume}
			Quelques conseils pour les petits premières potentiellement en train de flipper.
		\end{parchResume}
		Cher petit première, si tu stresses à l'idée de cette première semaine de semestrielles, rassure-toi : c'est loin d'être fini. Voici quelques conseils pour réussir cette première des quinze semaines de torture imposées par la maturité.
		
		\textbf{1 -- À l'avance, tu t'y prendras}\\
		Et oui les révisions, ça se fait à l'avance. Fais-toi un programme de révision et tiens-le.
		
		\textbf{2 -- Organisé(e) tu seras}\\
		(Contrairement à moi)
		
		\textbf{3 -- Concentré(e) tu seras}\\
		N'est-ce pas la clé pour réussir ?
		
		\textbf{4 -- Tôt tu te coucheras}\\
		Parce qu'une dissert' de 4h après une nuit de 2h c'est moyen.
		
		\textbf{5 -- La fête tu ne feras pas}\\
		Un oral en pleine gueule de bois c'est tendu… Bon, ça ne te dispense pas de venir à la fête de l'Escalade, parce que c'est la journée de l'année et qu'il paraît que les organisateurs sont cool
		
		\textbf{6 -- La biblio du collège tu éviteras}\\
		Et oui, quand on y va, on parle plus qu'on ne révise… Tentant, mais pas très efficace
		
		\textbf{7 -- Relaxé(e) tu seras}\\
		Ne stresse pas, c'est pas si difficile la matu, la preuve, on est encore tous là. Et puis en première, estimez-vous heureux, vos trimes ne comptent même pas pour la moitié du semestre…
		
		\textbf{8 -- Déterminé(e) tu resteras}\\
		Et oui, après cette semaine, il te restera 14 semaines de trimes à tenir. Sans compter les examens de matu.
		
		\textbf{9 -- Plein d'espoir tu resteras}\\
		Les miracles existent, et oui, tu peux avoir ta moyenne sans avoir lu le livre, mais bon, c'est pas conseillé.
		
		\textbf{10 -- À la fin, d'énormes vacances tu passeras}\\
		Parce que si t'as bien bossé, c'est complètement mérité !
	\end{parchArticle}
	
	\begin{parchArticle}{Le miroir noir}
		{Hervé}{De l'autre côté du miroir}{1}
		Ha ! Les semestrielles ! Ces charmantes deux semaines composées majoritairement de stress, de tests et d'une pincée de révisions. Or si comme moi tu fais partie de la très fameuse espèce du Collégius Glanditus, la relecture de tes cours risque beaucoup de se composer de 10 minutes de travail suivies immédiatement d'un épisode de ta série préférée, histoire d’aérer un peu tous ces neurones. Seulement là, une question se pose : « Mais qu'est-ce que je vais regarder moi?! ». Difficile de trouver une bonne série alors que la saison 6 de Game Of Thrones se fait attendre et qu'on a déjà vu l'ensemble de Breaking Bad pendant l'été. Mais ne t’inquiète pas, le Parchemin est là pour t'aider. En deux mots : Black Mirror.
		
		Black Mirror ou le miroir noir (par pitié, n'utilise jamais ce nom) est donc une série télévisée produite de l'autre côté de la manche par nos sympathiques cousins britanniques. Au menu : notre rapport à la technologie et les dérives qui pourraient en découler. Ça peut sembler très ennuyant de premier abord mais je vous le promets, cette série vaut le détour. Entre histoires complètement wtf, ironie acide et critique acerbe de la société, Black Mirror est de loin une des séries les plus intéressantes et amusantes qu'il m'ait été donné de voir. On y découvre ainsi comment un ours bleu virtuel particulièrement vulgaire va devenir le maître incontesté du monde ou encore comment la presse réagirait si on sommait le premier ministre britannique d'avoir des relations sexuelles avec une truie pour éviter la mort de la princesse Susannah, prise en otage par de mystérieux ravisseurs.
		
		Bon soyons honnêtes, la série n'est pas vraiment faite pour les plus délicats d'entre vous. Si on y trouve aucune image particulièrement choquante ou violente, la série reste crue sur beaucoup d'aspects. À réserver aux plus avertis donc. Cependant si le politiquement non correct ne te fais pas peur, je te dirais une seule chose : fonce, c'est de la bonne.
	\end{parchArticle}
	
	\newpage
	
	\begin{parchArticle}{Le jeu des oraux}
		{Vestin et Juliette}{Plongés dans leurs livres}{1}
		Amis collégiens, peu importe votre degré ou votre OS, vous n’êtes pas sans savoir que les semestrielles arrivent à grands pas. Mais rassurez-vous, pour vous redonner le sourire, nous vous proposons un petit jeu intitulé « paye ton oral ». Le principe est d’essayer de placer les phrases, mots ou expressions ci-dessous dans un examen oral ou dans une dissertation. Chaque phrase vous rapporte un certain nombre de points, celui qui en a le plus a gagné.  À vous de jouer!
		
		\noindent\begin{itemize}[leftmargin=*]
			\item Mangouste de Papouasie intergalactique (5)
			\item C’est ballot ! Ça va le chalet ? (4)
			\item Hitler faisait fureur (8)
			\item Pladimir Voutine (6)
			\item On n’est pas à la foire à la saucisse ! (4)
			\item C’est la mer noire (3)
			\item Comment est votre blanquette ? (7)
			\item Selon la congolexicomatisation des lois du marché (18)
			\item J’ai toujours aimé ce qui est oral (15)
			\item C’est comme chercher un oral dans une botte de foin (9)
			\item La guerre froide s’est déroulée en hiver (6)
			\item une gloriette absconse (10)
			\item La métaphysique de Descartes est totalement inspirée par le film Inception (12)
			\item Nous pouvons comparer Voltaire à Kim Kardashian (10)
		\end{itemize}
		\vspace*{\baselineskip}
		\textbf{Le Parchemin décline toute responsabilité en cas de mauvaise note.}
	\end{parchArticle}
	
	\parchTitreArticle{Phrases Boarf’ 3}
	\begin{multicols}{3}
		\begin{spacing}{.95}
%		\begin{parchResumeTheme}{Poésie moderne \& Relations sociales}
%			Do not read the next sentence. You, little rebel, I like you.
%		\end{parchResumeTheme}
		\noindent Chères lectrices, chers lecteurs, quel plaisir de vous retrouver en ce mois de décembre glacial pour la troisième édition de cette rubrique! Comme chaque mois, je vous ai concocté une sélection des plus fines phrases d’approche/de drague réparties par catégories. Approchez donc et réchauffons-nous auprès du feu ardent de la passion qui brûle au sein des perles de poésie moderne qui vont suivre.
		
		\textbf{Spéciales “Awww...” 3}\\
		-- Tu es comme un dictionnaire, tu ajoutes un sens à ma vie.\\
		-- Les gens pensent que Disneyland est le meilleur endroit du monde. On voit qu’ils n’ont pas été entre tes bras.\\
		-- Quand je suis avec toi, les heures ressemblent à des secondes, mais lorsque nous sommes séparés, les jours sont des années.\\
		-- Peux-tu m’apprendre à nager? Parce que je me noie dans tes yeux.\\
		-- La première fois que je t’ai vue, j’ai cherché une signature, parce que tous les chefs-d’œuvre en ont une.\\
		-- Tu crois en l’amour au premier regard ou je dois repasser devant toi?\\
		-- Es-tu un appareil photo? Parce que chaque fois que je te regarde, je souris.\\
		-- T’es un peu comme Netflix, parce que je peux te regarder des heures durant.\\
		-- Je cherchais une définition de la Beauté, mais je crois que je viens de la trouver.\\
		-- J’arrêterai de t’aimer le jour où une mangue poussera sur un pommier un 30 février.\\
		-- Je peux t’emprunter un baiser? Promis, je te le rends.\\
		-- Tes lèvres ont l’air seules, voudraient-elles rencontrer les miennes?\\
		-- Tu me rends plus heureux que lorsque je regarde mon réveil et que je me rends compte qu’il me reste deux heures de sommeil.\\
		-- Si je devais choisi entre respirer et t’aimer, j’utiliserai mon dernier souffle pour te dire que je t’aime.
		
		\vspace*{3\baselineskip}
		\parchCitation{Est-ce que je peux prendre une photo de toi? C’est pour montrer au Père Noël ce que je veux comme cadeau.}
		\vspace*{3\baselineskip}
		
		\textbf{Spéciales Pré-requis 2}\\
		-- [Regarder l’étiquette de l’un de ses habits][Quand il/elle demande ce que vous faites] Yep, c’est bien ce que je pensais: Made in Paradise / Fabriqué au paradis!\\
		-- Laisse-moi te lire ton futur … [Prendre sa main et noter son numéro dessus]... tu vas m’appeler ce soir.\\
		-- [Lorsqu’elle sourit] Un sourire ne peut pas changer le monde, mais ton sourire change le mien.
		
		\textbf{Spéciales Anglysche}\\
		-- I like you a lottle, it’s like a little, except a lot.\\
		-- [Au restaurant] Know what’s on the menu? Me - N - U.\\
		-- People say I am a superhero. Guess my name. I am \enquote{YourMan}.\\
		-- You wanna know what’s beautiful? Read the first word again.\\
		-- Your hands look heavy, can I hold it for you?\\
		-- I’ll buy you dinner, if you make me breakfast.\\
		-- People call me [your name], but you can call me Tonight.\\
		-- You look pretty. I look pretty. Why don’t we go home and stare at each other?\\
		-- Kissing is the language of love, so how about a conversation?
		
		C’est sur ces magnifiques paroles que je vous souhaite de bonnes semestrielles (bonne chance, on survivra \verb|^^|), de joyeuses fêtes de fin d’année, de très bonnes vacances, tout de bon pour cette année 2016, tous mes vœux de bonheur, de réussite, de... voire même bon anniversaire au point où on en est. À l’année prochaine, d’ici là portez-vous bien. Peace ;)
		
		\vspace*{1mm}
		\begin{mdframed}[
			skipabove=0mm,
			skipbelow=0mm,
			innerleftmargin=2ex,
			innerrightmargin=2ex,
			innertopmargin=2ex,
			innerbottommargin=2ex,
			rightline=false,
			topline=false,
			bottomline=false,
			linewidth=1mm,
			linecolor=mediumGrey,
			backgroundcolor=lightGrey,
			fontcolor=black,
			]
			\textbf{Renat}\\
			With Cupidon \& Baudelaire, mes gars sûrs
		\end{mdframed}
		
		\end{spacing}
	\end{multicols}
	
	\newpage
	
	\begin{parchArticle}{Retour dans le passé}
		{Elodie et Julia}{Mélancoliques}{1}
		\noindent Votre souvenir le plus marquant, que ce soit de l’école primaire, de la crèche ou d’un camp? Ce mois-ci, nous avons recueilli quelques anecdotes:
		
		\enquote{On a gagné un jeu d’équipe et on a même pas eu le kinder…}, nous a confié un jeune homme de 18 ans, en faisant la moue.
		
		\enquote{J’étais allée dormir chez mes grands-parents et le lendemain, j’avais pas envie de mettre les habits que mes parents avaient préparés et du coup, je suis allée à l’école en pyjama Barbie et c’était le jour de la photo de classe.}
		
		Le classique : \enquote{Je suis rentré chez moi en pantoufles sans m’en rendre compte.}
		
		Une jeune fille de 16 ans nous décrit le plus grand rêve de ses premières colos: \enquote{J’étais déguisée en princesse à une boom, et là, y’a mon n’amoureux qui est venu et… il m’a fait un bisou sur la joue *-*.}
		
		Mauvais souvenir : \enquote{Je suis resté bloqué dans l’ascenseur le dernier jour de l’école primaire.}
		
		\enquote{J’avais 12 ans et avec ma copine, on a voulu faire comme dans une vidéo. On a pris du déo et un briquet, on a enflammé le spray du déo et on a brûlé la porte d’une classe.}
		
		L’horreur : \enquote{Je jouais à cache-cache et je me suis caché dans une poubelle, mais je suis resté coincé et on a dû appeler les pompiers pour m’aider à sortir. Après, ils ont changé toutes les poubelles de l’école.}
		
		\enquote{J’étais à la crèche et on devait souffler dans une paille qui était trempée dans de l’eau et du savon pour faire des bulles. J’avais pas compris le principe, j’ai aspiré au lieu de souffler. Laissez tomber j’suis un cassos.}
		
		Et vous ? Quel est votre souvenir le plus mémorable ?
	\end{parchArticle}
	
	\begin{parchArticle}{Hi-han}
		{Renat}{Au fond du trou, mais creuse encore}{1}
		\begin{parchResumeTheme}{Conte}
			Une petite histoire décalée avec un fond de vérité…
		\end{parchResumeTheme}
		Un jour, l’âne d’un fermier tomba dans un puits. L’animal gémit pitoyablement pendant des heures et le fermier se demandait bien ce qu’il allait faire. Finalement, il se rappela que l’animal était vieux et que, de toute façon, le puits devait disparaître. Il en conclut donc qu'il n’était pas rentable de tenter de récupérer l’âne.
		
		Il appela tous ses voisins et leur demanda de venir l’aider. Chacun saisit une pelle et ils commencèrent à combler le puits. Au début, l’âne, réalisant ce qui se produisait, se mit à crier terriblement. Puis, à la stupéfaction de tout le monde, il se tût. Quelques pelletées plus tard, poussé par la curiosité, le fermier regarda finalement dans le fond du puits et fut étonné…
		
%		\parchCitation{La vie est belle!}
%		\vspace*{-\baselineskip}
		
		A chaque pelletée de terre qui tombait sur lui, l’âne réagissait aussitôt : il se secouait pour enlever la terre de son dos et piétinait ensuite le sol sous ses sabots. Pendant que les voisins du fermier continuaient à jeter de la terre et des cailloux sur l’animal, il se secouait et montait toujours plus haut. Bientôt, tous furent stupéfaits de voir l’âne sortir du puits et se mettre à trotter !
		
		La vie essaye parfois de vous « engloutir sous toutes sortes d’ordures » et de décombres. L’astuce pour « se sortir du trou » est de « se secouer pour mieux avancer » ; chacun de nos ennuis, nos épreuves ou échecs est une pierre qui permet de progresser. Nous pouvons sortir des puits les plus profonds en n’arrêtant jamais de nous battre. Ne vous laissez pas abattre, ne baissez jamais les bras, vous pouvez toujours remonter ! C’est possible ! Et n’oubliez pas : « possible » ne veut pas dire « facile ». Mais ce n’est pas parce que c’est difficile que c’est impossible ! La vie est parfois difficile, oui, mais les batailles  valent la peine d'être menées ! La vie est belle ! Simplifiez-vous la vie et soyez heureux !
	\end{parchArticle}
	
	\vspace*{\baselineskip}
	\parchTitreArticle{\hfill Merci à toute l'équipe\hfill}
	\begin{multicols}{2}
		\noindent\textbf{Rédacteurs en chef:}\\
		Lino Mercolli et Yves Zumbach\\
		\textbf{Rédacteurs:}\\
		Hervé Zumbach, Renat Arjantsev, Élodie Ducrest, Chiara Miele, Damien Geissbuhler, Eliott Wiggins, Juliette Wiggins, Oriane Rutsche, Valentina Scariati, Vestin Hategekimana, Julia Herbert, Yann Alhadeff, Garance Sallin, Yves Zumbach\\
		\textbf{Illustratrices:}\\
		Alina Staub, Candice Vezza, Melina Alhadeff\\
		\noindent\textbf{Correctrices:}\\
		Garance Sallin et Mathilde Genoud\\
		\textbf{Mise en page:}\\
		Yves Zumbach
	\end{multicols}
	
	\newpage
	
	\parchTitreArticle{Mots-croisés}
	
	\vspace*{\baselineskip}
	\noindent\begin{minipage}{.2\textwidth}
		\noindent\textbf{Vertical}\\
		\maDef{v}{Tyrannie ou malaise}\\
		\maDef{v}{Jeu où l’on finissait au ciel / Musique de film }\\
		\maDef{v}{Qui suscite l’aversion}\\
		\maDef{v}{Mousse blanche / Origine}\\
		\maDef{v}{Élément 38 / Vent froid et sec qui peut être gagnant}\\
		\maDef{v}{Identité/ Vaillant}\\
		\maDef{v}{Interjection utilisée par les videurs de boîtes / 6}\\
		\maDef{v}{Définit les sorcières}\\
		\maDef{v}{Ainsi / Résistance qui chauffe}\\
		\maDef{v}{Comprendre ou créer}
		
		\vspace*{1em}
		\noindent\textbf{Horizontal}\\
		\newlist
		\maDef{h}{Mensonges qui ne sont pas des mensonges}\\
		\maDef{h}{Unité de pression/ Qui est difficile / Choix à faire en fin de 2ème}\\
	\end{minipage}\hspace*{-5mm}\begin{minipage}{.6\textwidth}
		\begin{gridenv}
			\centering
			\tikzstyle{gridstyle}=[thick]
			\noindent\begin{tikzpicture}[scale=1.8, every node/.style={scale=1.5}] \hspace{-4mm}
			\begin{crossgrid}[h=10,v=10]
			\blackcases{10/1, 3/2, 8/2, 5/3, 6/3, 9/3, 7/5, 4/6, 4/7, 7/7, 2/8, 4/8, 7/8, 3/9, 6/9} 
			\end{crossgrid}
			\end{tikzpicture}
		\end{gridenv}
	\end{minipage}
	\begin{minipage}{.2\textwidth}
	\noindent\hspace*{-1ex}\maDef{h}{Raccourci de programme/ Sans Valeur}\\
	\maDef{h}{Rachat spirituel}\\
	\maDef{h}{A causé la perte de Tristan et Iseult/ Permet de voir les atomes}\\s
	\maDef{h}{Service des Loisirs Éducatifs/ Épice qui ouvre toutes les portes}\\
	\maDef{h}{Sureau en ancien français / Sans voix / Sept pour un romain qui a bu}\\
	\maDef{h}{Rayon X / Norme Quantitative d’Obésité }\\
	\maDef{h}{Souhait allemand / Ce que dit un russe assoiffé quand on lui propose un verre de vodka / Début du verbe vouloir sous sa forme polie}\\
	\maDef{h}{Écrire en roman}\\
	\hfill\textbf{Par la Chiure de gomme}
\end{minipage}
	
	\vspace*{.3\baselineskip}
	\noindent\mbox{}
	\noindent\begin{minipage}[b]{.33\textwidth}
		\noindent\mbox{}
		\parchTitreArticle{Sudoku}
		\vspace*{\baselineskip}
		
		\noindent \textbf{Niveau : facile}
		
		\noindent Remplissez la grille en fonction des règles de base du sudoku: Les chiffres 1 à 9 figurent obligatoirement une seule fois sur chaque ligne, chaque colonne et chaque carré 3x3
		
		\noindent\null\hfill\textbf{Par Eliott}
		
		\vspace*{4.2cm}
		\parchTitreArticle{Bonnes\\[1mm]vacances!!}
		\vspace*{-4mm}
	\end{minipage}\hspace*{4mm}
	\begin{minipage}[b]{.73\textwidth}
		\noindent\mbox{}
		\noindent\begin{tikzpicture}[scale=1.22]
		\draw (0, 0) grid (9, 9);
		\draw[very thick, scale=3] (0, 0) grid (3, 3);
		
		\setcounter{row}{1}
		\setrow { }{5}{7}  { }{3}{8}  {6}{ }{2}
		\setrow { }{4}{ }  { }{ }{ }  { }{ }{1}
		\setrow { }{ }{ }  {7}{ }{9}  {5}{ }{4}
		
		\setrow {3}{ }{8}  {2}{ }{6}  {4}{5}{7}
		\setrow { }{ }{ }  {1}{ }{ }  { }{2}{6}
		\setrow {9}{ }{6}  { }{4}{ }  {8}{ }{ }
		
		\setrow { }{3}{ }  { }{ }{2}  {7}{6}{9}
		\setrow {6}{ }{5}  { }{ }{ }  { }{ }{ }
		\setrow {7}{ }{2}  { }{ }{ }  {1}{3}{ }
		\end{tikzpicture}
	\end{minipage}

\end{document}
